%============================================================
% PHY2027 Assessment 3 – Final Project Report
% Option 3: 2D Ising Model Simulation
%============================================================
\documentclass[10pt,a4paper]{article}
\usepackage[margin=2.5cm]{geometry}
\usepackage{amsmath,amssymb}
\usepackage{graphicx}
\usepackage{caption,subcaption}
\usepackage{booktabs}
\usepackage{listings}
\usepackage{xcolor}
\usepackage{fancyhdr}
\usepackage{inconsolata}

% Code listing style
\lstset{
    language=C,
    basicstyle=\ttfamily\small,
    keywordstyle=\color{blue}\bfseries,
    commentstyle=\color{gray},
    numbers=left,
    numbersep=8pt,
    frame=single,
    captionpos=b,
    breaklines=true
}

% Header/Footer
\pagestyle{fancy}
\fancyhf{}
\rhead{PHY2027 – Ising Model Report}
\lhead{Miguel de Sousa}
\cfoot{\thepage}

\title{Two-Dimensional Ising Model Simulation\\
       \large PHY2027 – Scientific Programming in C – Final Project Report}
\author{Miguel de Sousa}
\date{\today}

\begin{document}

\maketitle

\section{Theory}
[1] The Ising model, introduced by Ernst Ising in 1925, is a mathematical model consisting of a lattice of 'spin' variables $\sigma_\alpha$ which can take discrete values of $\pm1$.
These spins represent magnetic dipole moments of atomic spins that can be in one of two states: up (+1) or down (-1). The model is used to study phase transitions and critical phenomena in statistical mechanics.
[1]Any two of these spins have a mutual interaction Energy
\begin{equation}
    -E_{\alpha\beta}\sigma_\alpha\sigma_\beta
\end{equation}

[1]Additionally, an external magnetic field $B$ can interact with each spin, contributing an energy term of
\begin{equation}
    -B\sigma_\alpha
\end{equation}

The total energy of a configuration of spins on a lattice can be expressed as
\begin{equation}
    E = -\sum_{\langle \alpha,\beta \rangle} E_{\alpha\beta}\sigma_\alpha\sigma_\beta - B\sum_{\alpha} \sigma_\alpha
\end{equation}
where the first sum is over all pairs of neighbouring spins.
The Ising model exhibits a phase transition at a critical temperature $T_c$, below which the spins tend to align, resulting in spontaneous magnetisation, and above which the spins are disordered.

\section{Implementation}
\subsection{Data Structure and Initialisation (Part a)}
I began by defining a structure type to represent the 2D Ising lattice, which included all necessary parameters as structure members for the model such as its N-dimenionsionality, values for temperature T, external magnetic field B, total energy and magnetisation of the lattice, and a pointer to an array of spins.

\begin{lstlisting}
typedef char Spin;
static const Spin SPIN_UP   =  1;
static const Spin SPIN_DOWN = -1;

typedef struct{
    int N;
    Spin *spin;
    double T;
    double B;
    unsigned long long step;
    double energy;
    double magnetisation;
} IsingLattice;
\end{lstlisting}

Lines 1-3: Assigning a char type to Spin, with constants for SPIN\_UP and SPIN\_DOWN defined with this new type. 
The char data type forces each spin to occupy only 1 byte of memory, which is efficient for large n by n lattices. 
The Spin type is used throughout the code, so defining it at the start improves code readability.\medbreak

Lines 5-14: Creating the IsingLattice structure type with members for lattice size N, pointer to spin array, temperature T, external magnetic field B, current step number, total energy, and total magnetisation.
Structures are useful for grouping related data together, so its perfect for representing the lattice and its properties.
I used double for T, B, energy, and magnetisation to allow for fractional values and higher precision in calculations.
The step member is an unsigned long long to allow a large number of Monte Carlo steps.
Proving its reuseability, the spin pointer defined in the sructure types is allocatd to my new Spin data type.\medbreak

After defining the structure type, I implemented the initialisation function to set up the lattice with random n by n spins.

\begin{lstlisting}
IsingLattice *create_lattice(int N, double T, double B){

    IsingLattice *lattice = malloc(sizeof(IsingLattice));

    if (lattice == NULL){
        fprintf(stderr, "Memory allocation failed\n");
        exit(EXIT_FAILURE);
    }

    lattice->N = N;
    lattice->T = T;
    lattice->B = B;
    lattice->step = 0;
    lattice->spin = malloc(N * N * sizeof(char));

    if (lattice->spin == NULL){
        fprintf(stderr, "Memory allocation for spin failed\n");
        free(lattice);
        exit(EXIT_FAILURE);
    }

    for (int i = 0; i < N * N; i++){
        lattice->spin[i] = (rand() % 2) ? SPIN_UP : SPIN_DOWN;
    }

    return lattice;
}
\end{lstlisting}

Lines 1-8: The create\_function takes lattice parameters as arguments and it allocates enough memory for the new lattice instance using malloc
and the size of the IsingLattice structure type.
It checks if the allocation was successful, printing an error message and exiting if not.
The structure type pointer is initalised within the fucntion as *lattice.\medbreak

Lines 10-13: Assign the input parameters N, T, and B to the corresponding structure members in the IsingLattice using -> structure pointer notation.\medbreak

Lines 14-24: Allocate memory for the spin array within the lattice structure using malloc.
The size is n by n multiplied by the size of a char (1 byte) since each spin is represented as a char.
It checks for successful allocation, freeing the previously allocated lattice memory location if the spin allocation fails.
I then assigned each lattice spin array value in the lattice to a random value of SPIN\_UP or SPIN\_DOWN using a count controlled loop with boundaries at 0 and n by n.
The rand() function acted as the randomiser, with modulus 2 giving eitehr values of 0 or 1, which were mapped to either SPIN\_UP or SPIN\_DOWN.\medbreak

Lines 26: Finally, the function returns a pointer to the newly created and initialised lattice instance.


\subsection{Energy and Magnetisation Calculation (Part a)}
\subsection{Metropolis Algorithm (Part b)}
\subsection{Visualisation Method (Part c)}

\section{Program Exploration and Results}
\subsection{Zero External Field ($B=0$)}
\subsection{Effect of Temperature}
\subsection{Effect of External Magnetic Field ($B \neq 0$)}
\subsection{Boundary and Neighbour Effects}

\section{Conclusion}

\begin{thebibliography}{1}
\bibitem{mccoy2014}
B.~M.~McCoy and T.~T.~Wu, \textit{The Two-Dimensional Ising Model}, Harvard University Press, Reprint 2014.
\end{thebibliography}

\end{document}